\documentclass[a4paper]{report}
\usepackage[utf8]{inputenc}
\usepackage[portuguese]{babel}
\usepackage{a4wide}
\usepackage{hyperref}
\hypersetup{pdftitle={Programação em logica estendida e Conhecimento imperfeito},
pdfauthor={Joao Teixeira, Jose Filipe Ferreira, Miguel Solino},
colorlinks=true,
urlcolor=blue,
linkcolor=black}
\usepackage{subcaption}
\usepackage[cache=false]{minted}
\usepackage{listings}
\usepackage{booktabs}
\usepackage{multirow}
\usepackage{appendix}
\usepackage{tikz}
\usepackage{authblk}
\usepackage[parfill]{parskip}
\usetikzlibrary{positioning,automata,decorations.markings}

\begin{document}

\title{Programação em lógica estendida e Conhecimento imperfeito \\
\large Grupo 7}
\author{João Teixeira (A85504) \and Jose Filipe Ferreira (A83683) \and Miguel
Solino (A86435)}
\date{\today}

\begin{center}
    \begin{minipage}{0.75\linewidth}
        \centering
        \includegraphics[width=0.4\textwidth]{eng.jpeg}\par\vspace{1cm}
        \vspace{1.5cm}
        \href{https://www.uminho.pt/PT}
        {\color{black}{\scshape\LARGE Universidade do Minho}} \par
        \vspace{1cm}
        \href{https://www.di.uminho.pt/}
        {\color{black}{\scshape\Large Departamento de Informática}} \par
        \vspace{1.5cm}
        \maketitle
    \end{minipage}
\end{center}

\begin{abstract}
    \begin{center}
        O objetivo deste projeto é desenvolver um sistema de representação de
        conhecimento e raciocínio, com recurso à linguagem de programação em
        lógica \textit{Prolog}.
    \end{center}
\end{abstract}

\tableofcontents

\pagebreak

\chapter{Introdução}
Este projeto tem como objetivo a criação de um sistema de representação de
conhecimento e raciocínio que consiga representar informação sobre contratos
públicos.

Para a realização deste trabalho utilizamos a linguagem de programação em lógica
lecionada nesta UC, \textit{Prolog}.

Em primeiro lugar, iremos apresentar em detalhe o problema proposto, identificar
os requisitos e que funcionalidades implementar para responder aos requisitos
determinados.

De seguida explicaremos a solução encontrada para este problema, entrando em
detalhe sobre as representações de conhecimento escolhidas, diversos
invariantes, como é efetuada a evolução e regressão do conhecimento, entre
outros, entre outros..

\chapter{Problema}

O programa tem que cumprir os seguintes requisitos:
\begin{itemize}
        \item Representar conhecimento positivo e negativo;
        \item Representar casos de conhecimento imperfeito, com recurso à
            utilização de valores nulos imprecisos, interditos e incertos;
        \item Manipulação de invariantes, com o intuito de manter a consistência
            do conhecimento na base de conhecimento;
        \item Permitir a evolução e regressão do conhecimento existente;
        \item Possuir um sistema de inferência capaz de seguir os mecanismos de
            raciocínio inerente ao sistema.
\end{itemize}

\chapter{Solução}

\section{Predicados Implementados}

Para a implementação deste sistema, foram criados três predicados para
representar a informação do sistema
\begin{itemize}
    \item adjudicante: NifAd, Nome, Morada $\rightarrow$ \{V, F, D\};
    \item adjudicária: NifAda, Nome, Morada $\rightarrow$ \{V, F, D\};
    \item contrato: \#Id, NifAd, NifAda, TipoDeContrato,
        TipoDeProcedimento, Descrição, Valor, Prazo, Local, Dia, Mês, Ano 
        $\rightarrow$ \{V, F, D\};
\end{itemize}

Como estamos perante um contexto de lógica estendida, os predicados podem ter
valor de Verdadeiro, Falso, ou Desconhecido.

\section{Representação do conhecimento}

\subsection{Conhecimento Positivo}
O conhecimento positivo é representado através da inserção simples na base de
conhecimento como nos exemplos abaixo:

\begin{verbatim}
adjudicária(500000000, sarilhos, braga).
adjudicária(500000001, limpalimpinho, porto).
adjudicária(500000002, oficina, lisboa).

adjudicante(600000000, bdp, lisboa).
adjudicante(600000001, chmtad, vilareal).

contrato(0, 600000000, 500000000, "Aquisicao de servicos", ad, 
    "Assessoria juridica", 4000, 100, porto, 10, 10, 2010).
contrato(1, 600000003, 500000000, "Aquisicao de servicos", cprev, 
    "Assessoria juridica", 1000000, 346, porto, 10, 10, 2010).
\end{verbatim}

\pagebreak
\subsection{Conhecimento Negativo}
Com a presença de conhecimento imperfeito, faz sentido a introdução de
conhecimento negativo. Este pode ser obtido representado através de duas formas,
negação explícita, e também a negação forte.

A negação explícita é representada através da inserção simples na base de
conhecimento como no exemplo abaixo:

\begin{verbatim}
-adjudicante(600000010, bombeiros, evora).
\end{verbatim}

Neste exemplo corresponde a dizer que não existe um adjudicante com o Nif
600000010, com o nome "bombeiros" e sede em Évora.

A inserção deste tipo de conhecimento corresponde a adicionar o prefixo
\textit{-} a um predicado, fazendo-o assim ser a negação deste.

A negação forte deriva da aplicação do Pressuposto do Mundo Fechado, isto é, o
conhecimento que não tem prova de ser verdadeiro, é implicitamente falso. 
Para a implementação deste pressuposto, procedemos à extensão do predicado
\textit{-T}, como representado abaixo, onde \textit{T} corresponde ao predicado
de \textit{adjudicante}:

\begin{verbatim}
-adjudicante(Nif, Nome, Morada) :-
    not(adjudicante(Nif, Nome, Morada)),
    not(excecao(adjudicante(Nif, Nome, Morada))).
\end{verbatim}

Como visto no exemplo, o conhecimento será interpretado como negativo caso não
seja positivo, nem seja uma exceção, o como abaixo veremos, será interpretado
como desconhecido.

\subsection{Conhecimento Imperfeito Incerto}
O conhecimento imperfeito incerto corresponde a conhecimento ao qual não temos a
certeza, nem temos ideia do que poderá ser. Para a representação deste,
inserimos o facto na base de conhecimento, com o campo o qual não temos a
certeza com um valor arbitrário, uma exceção que será correspondida sempre que
se tente aceder ao conhecimento, e inserimos na base de conhecimento a
informação do campo nulo.

Abaixo, está presente um exemplo da representação deste tipo de conhecimento,
neste caso sendo correspondente à adjudicária com Nif 500000003, com sede em
Santarém, da qual não sabemos o nome.
\begin{verbatim}
adjudicataria(500000003, incerto, santarem).
excecao(adjudicataria(Nif, _, Morada)) :-
    adjudicataria(Nif, incerto, Morada).
incertoNome(adjudicataria(500000003), incerto).
\end{verbatim}

\subsection{Conhecimento Imperfeito Impreciso}
O conhecimento imperfeito impreciso consiste no conhecimento desconhecido dentro
de um conjunto finito de hipóteses. Para a representação deste fazemos a
inserção de exceções com as diversas hipóteses, inserimos também o facto do Id
corresponder a conhecimento impreciso.

No exemplo abaixo podemos ver a representação do adjudicante com Nif 600000005,
e sede em Lisboa, o qual não sabemos se corresponde aos Bombeiros ou ao
Exercito.

\begin{verbatim}
excecao(adjudicante(600000005, exercito, lisboa)).
excecao(adjudicante(600000005, bombeiros, lisboa)).
impreciso(adjudicante(600000005)).
\end{verbatim}

\subsection{Conhecimento Imperfeito Interdito}

O conhecimento interdito corresponde ao conhecimento que nos é desconhecido, mas
nunca vai ser possível descobrir.

Para a representação deste conhecimento, procedemos à introdução do predicado
com um valor arbitrário no campo desconhecido, da exceção que vai ser
correspondida quando quisermos aceder ao campo desconhecido, da informação que o
campo corresponde a um valor interdito, e por fim, de um invariante que garante
que não é inserido novo conhecimento com que não tenha o campo interdito.

Como exemplo, apresentamos abaixo o contrato com Id 5, o qual não é possível
saber o adjudicante.

\begin{verbatim}
contrato(5, 600000001, interdito, "Aquisicao de servicos", cprev, 
    "Assessoria juridica", 2000, 520, porto, 10, 10, 2010).
excecao(contrato(Id, Ad, _, TContrato, TProcedimento, Descricao, 
    Custo, Prazo, Local, Dia, Mes, Ano)) :-
        contrato(Id, Ad, interdito, TContrato, TProcedimento, 
            Descricao, Custo, Prazo, Local, Dia, Mes, Ano).
interditoAda(contrato(5), interdito).
+contrato(_, _, _, _, _, _, _, _, _, _, _, _) :: (findall(Ada, 
        (contrato(5, _, Ada, _, _, _, _, _, _, _, _, _),
        not(interditoAda(contrato(5), Ada))), S),
    length(S, N),
    N == 0).
\end{verbatim}

\section{Evolução do conhecimento}

\subsection{Evolução do conhecimento perfeito}

Para a evolução do conhecimento perfeito temos duas hipóteses, quando já existe
conhecimento imperfeito referente ao predicado em questão na base de
conhecimento, ou quando é a primeira informação que inserimos sobre o predicado.

Na primeira hipótese, para a atualização do conhecimento imperfeito, primeiro é
verificado se existe informação sobre conhecimento interdito, que como não é
possível saber, não é possível atualizar este predicado. Caso não exista
referência a conhecimento interdito, é removido o conhecimento imperfeito e
adicionado o conhecimento perfeito. No exemplo abaixo, é possível observar como
é efetuado a atualização do conhecimento de um adjudicante. 

\begin{verbatim}
evolucao(adjudicante(Nif, Nome, Morada)) :-
    not(interditoNome(adjudicante(Nif), _)),
    not(interditoMorada(adjudicante(Nif), _)),
    removeIncerto(adjudicante(Nif)),
    inserir(adjudicante(Nif, Nome, Morada)).
\end{verbatim}

Caso não haja informação presente para o predicado a adicionar, é apenas feita a
inserção na base de conhecimento, como visível abaixo.

\begin{verbatim}
evolucao(Termo) :-
    demo(Termo, falso),
    inserir(Termo).
\end{verbatim}

\subsection{Evolução do conhecimento imperfeito Impreciso}

A evolução deste tipo de conhecimento é efetuada recebendo uma lista com as
hipóteses possíveis, sendo verificado se é fornecido uma lista com mais do que
um elemento, se todos os elementos da lista correspondem ao mesmo Id, com o
predicado \textit{mesmoNif}, se nenhum representa conhecimento 
anteriormente declarado como perfeito, com o predicado
\textit{nenhumPerfeito}, e só assim são inseridas as exceções 
correspondentes ao conhecimento fornecido, através do predicado
\textit{insereExcecoes}.

Como exemplo, incluímos o predicado responsável pela evolução do predicados
correspondentes a adjudicatárias.

\begin{verbatim}
evolucaoImpreciso([adjudicataria(Nif, N, M) | R]) :-
    R \= [],
    mesmoNif(R, Nif),
    nenhumPerfeito([adjudicataria(Nif, N, M) | R]),
    insereExcecoes([adjudicataria(Nif, N, M) | R]).
\end{verbatim}

\subsection{Evolução do conhecimento imperfeito Incerto}

A evolução deste tipo de conhecimento é efetuada, primeiro verificando que não
existe já conhecimento sobre o campo no qual a informação é incerta, depois
passa por verificar que o conhecimento não foi já marcado como interdito, e por
fim é inserido o conhecimento, respeitando os invariantes já determinados.

Como exemplo podemos ver o predicado que insere uma adjudicataria com nome
incerto.

\begin{verbatim}
evolucaoIncertoNome(adjudicataria(Nif, N, M)) :-
    demo(adjudicataria(Nif, _, M), falso),
    not(interditoMorada(adjudicataria(Nif), _)),
    assert(excecao(adjudicataria(_, _, _)) :- adjudicataria(_, N, _)),
    inserir(adjudicataria(Nif, N, M)),
    assert(incertoNome(adjudicataria(Nif), N)).
\end{verbatim}

\subsection{Evolução do conhecimento imperfeito Interdito}

A evolução deste tipo de conhecimento consiste em primeiro lugar, inserir o
predicado com o conhecimento interdito na base de conhecimento, respeitando os
invariantes definidos, e se estes forem respeitados, inserir os restantes
elementos necessários à representação deste tipo de conhecimento.

Para exemplo, apresentamos abaixo o predicado que permite a evolução de um
adjudicante que tem o nome interdito

\begin{verbatim}
evolucaoInterditoNome(adjudicante(Nif, Nome, Morada)) :-
    inserir(adjudicante(Nif, Nome, Morada)),
    assert(excecao(adjudicante(_, _, _)) :- adjudicante(_, Nome, _)),
    assert(interditoNome(adjudicante(Nif), Nome)),
    assert(+adjudicante(_, _, _) :: (findall(L, (adjudicante(Nif, L, _), not(interditoNome(adjudicante(Nif), L))), S), length(S, 0))).
\end{verbatim}

\section{Regressão do conhecimento}

\subsection{Regressão de conhecimento perfeito}

Para a regressão do conhecimento perfeito, é verificado se o predicado é
verdadeiro e, em seguida, é removido o conhecimento caso sejam respeitados todos
os invariantes.

\subsection{Regressão do conhecimento Imperfeito}

A regressão de conhecimento imperfeito é efetuada, verificando se o predicado
existe, no caso de conhecimento incerto, ou se a exceção correspondente, no caso
de conhecimento impreciso. Depois desta verificação, são encontrados todos os
invariantes que se aplicam, e depois removido o conhecimento e testado se os
invariantes ainda são respeitados. Caso não sejam, as alterações são revertidas.

\chapter{Sistema de inferência}

Para a implementação dos valores desconhecidos, foi preciso desenvolver um novo
sistema de inferência, capaz de lidar com valores verdadeiros, falsos e também
desconhecidos. Assim chegamos ao seguinte sistema de inferência: 

\begin{verbatim}
demo( Questao,verdadeiro ) :-
    Questao.
demo( Questao, falso ) :-
    -Questao.
demo( Questao,desconhecido ) :-
    not( Questao ),
    not( -Questao ).
\end{verbatim}

Aqui, caso o predicado seja verdadeiro, irá responder verdadeiro, se o predicado
for falso, ou seja, ser negado explicitamente, ou não haver informação que o
prove, a resposta será falsa, e se existir conhecimento imperfeito sobre o
predicado a avaliar, retornará desconhecido.

\section{Invariantes}

\subsection{Inserção e Remoção de conhecimento}

Para garantir a consistência do conhecimento presente na base de conhecimento,
foram implementados alguns invariantes, como por exemplo:

\begin{verbatim}
+adjudicante(Nif, _, _) :: (findall(Nif, (adjudicante(Nif, _, _)), S),
                                    length(S, N),
                                    N == 1).
\end{verbatim}

O invariante acima garante que apenas há um adjudicante associado a um Nif. De
forma análoga, foram implementados invariantes para as adjudicatárias e para
contratos e respetivos Ids.

Para garantir que os contratos tem informação válida, foi criado também um
invariante que garante que ambos os Nifs do Adjudicante e Adjudicatária existem
na base de conhecimento.

\begin{verbatim}
+contrato(_, Adjudicante, Adjudicataria, _, _, _, _, _, _, _, _, _) 
    :: (findall(Adjudicante, (adjudicante(Adjudicante, _, _)), Se),
        length(Se, Ne),
        Ne == 1,
        findall(Adjudicataria, (adjudicataria(Adjudicataria, _, _)), Sa),
        length(Sa, Na),
        Na == 1).
\end{verbatim}

Para evitar fraudes e destruição de informação sensível, incluimos um invariante
que impede a remoção de contratos.

\begin{verbatim}
-contrato(_, _, _, _, _, _, _, _, _, _, _, _) :: fail.
\end{verbatim}

De forma a garantir que não é possível invalidar informação de contratos já
criados, apenas é possível a remoção de adjudicantes e adjudicatárias que não
tenham contratos associados. No exemplo seguinte é apresentado o caso dos
adjudicantes, sendo o das adjudicatárias análogo.

\begin{verbatim}
-adjudicante(Nif, _, _) :: 
    (findall(Id, contrato(Id, Nif, _, _, _, _, _, _, _, _, _, _), S), 
    length(S, 0)).
\end{verbatim}

\subsection{Regra dos três anos}

Para implementar a regra dos três anos, foi criado um invariante que faz o
somatório do valor de todos os contratos entre um adjudicante e uma
adjudicatária, exceto o contrato a adicionar, e verifica se o valor total é
inferior a 75000 euros.

\begin{verbatim}
+contrato(Id, Ad, Ada, TC, TP, D, C, P, L, Dia, Mes, Ano) 
    :: (findall(Custo, (contrato(_, Ad, Ada, _, _, _, Custo, _, _, _, _, A),
                        A >= Ano - 3,
                        A =< Ano,
                        not(contrato(Id, Ad, Ada, TC, TP, D, C, P, L, Dia, Mes, Ano))),
                S),
            sum(S, R),
            R =< 75000).

\end{verbatim}

\subsection{Regras gerais de Contratos}

De forma a garantir que todos os contratos seguem os termos legais impostos, foi
implementado um invariante para validar todas as regras, que possui uma data
válida e o tipo de procedimento é válido. Para isso foi implementado o predicado
\textit{checkContract}, e o invariante verifica que não há nenhum contrato que
não verifique as condições.

\begin{verbatim}
+contrato(_, _, _, _, _, _, _, _, _, _, _, _) 
    :: (findall(Id, 
            (not(checkContract(contrato(Id, _, _, _, _, _, _, _, _, _, _, _)))),
            S),
        length(S, N),
        N == 0).
\end{verbatim}

\chapter{Conclusão}

Com este trabalho prático foi-nos possível consolidar conhecimentos sobre os
diversos tipos de conhecimento imperfeito, e por em prática os conceitos de
lógica estendida adquiridos no decorrer desta Unidade curricular. Posto isto
fazemos um balanço bastante positivo deste trabalho.

\end{document}
